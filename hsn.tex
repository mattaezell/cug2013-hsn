\documentclass[10pt, conference, compsocconf]{IEEEtran}

% Various packages that might be useful
\usepackage[pdftex]{graphicx}
\usepackage{array}
\usepackage[tight,footnotesize]{subfigure}
\usepackage{url}
\hyphenation{op-tical net-works semi-conduc-tor}
\usepackage{color}


\begin{document}

\title{Understanding the Impact of Interconnect Failures on System Operation}

\author{\IEEEauthorblockN{Matt Ezell}
\IEEEauthorblockA{High Performance Computing Operations\\
Oak Ridge National Laboratory\\
Oak Ridge, TN\\
ezellma@ornl.gov}
}

\maketitle

\begin{abstract}
Hardware failures are inevitable on large high performance computing systems.
Faults or performance degradations in the high-speed network can reduce the
entire system’s performance. Since the introduction of the Gemini interconnect,
Cray systems have become resilient to many networking faults. These new network
reliability and resiliency features have enabled higher uptimes on Cray systems
by allowing them to continue running with reduced network performance. Oak
Ridge National Laboratory has developed a set of user-level diagnostics that
stresses the high-speed network and searches for components that are not
performing as expected. Nearest-neighbor bandwidth tests check every network
chip and network link in the system. Additionally, performance counters stored
on the network ASIC’s memory mapped registers (MMRs) are used to get a more
full picture of the state of the network. Applications have also been
characterized under various suboptimal network conditions to better understand
what impact network problems have on user codes.
\end{abstract}

\begin{IEEEkeywords}
HPC; Cray; Gemini; Interconnect; HSN; Titan
\end{IEEEkeywords}

\section{Introduction}

\section{The Cray Gemini Network}

\subsection{3D Torus Topology}

\subsection{The Gemini Network Chip}
ptiles, ntiles, lane, channel, link, mezzanine, chassis/cage

\subsection{Fault Tolerence}
lane degrades, link inactives, quiesce and reroute

\subsection{Throttling}
care to talk about the c1 auto-throttles?

\subsection{Balanced Injection}

\section{The TopoBW Microbenchmark}

\subsection{Topology Awareness}
pmi, rca
knowing what kind of cable is in use

\subsection{MPI Messaging}

\subsection{Results from Titan}

\section{The Latitudes Microbenchmark}

\subsection{MMR Access}
ioctl and gpcd


\section{Real World Applications}

\subsection{S3D: Turbulent Combustion}

\subsubsection{S3D Introduction}



\section{Future Work}

\section{Conclusion}

% use section* for acknowledgement
%\section*{Acknowledgment}

%\textcolor{red}{
%	The authors would like to thank...
%	more thanks here
%}

% trigger a \newpage just before the given reference
% number - used to balance the columns on the last page
% adjust value as needed - may need to be readjusted if
% the document is modified later
%\IEEEtriggeratref{8}
% The "triggered" command can be changed if desired:
%\IEEEtriggercmd{\enlargethispage{-5in}}

% references section

% can use a bibliography generated by BibTeX as a .bbl file
% BibTeX documentation can be easily obtained at:
% http://www.ctan.org/tex-archive/biblio/bibtex/contrib/doc/
% The IEEEtran BibTeX style support page is at:
% http://www.michaelshell.org/tex/ieeetran/bibtex/
%\bibliographystyle{IEEEtran}
% argument is your BibTeX string definitions and bibliography database(s)
%\bibliography{IEEEabrv,../bib/paper}
%
% <OR> manually copy in the resultant .bbl file
% set second argument of \begin to the number of references
% (used to reserve space for the reference number labels box)
\begin{thebibliography}{1}

\bibitem{Walsh2009}
John Walsh, Troy Baer, Victor Hazlewood, Junseong Heo, Rick Mohr.
\newblock{Large Lustre File System Experiences at NICS}.
\newblock{Cray User Group 2009}.
\end{thebibliography}


% that's all folks
\end{document}


